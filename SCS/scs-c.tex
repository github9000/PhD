% Latex 'old' style.
\documentstyle[11pt]{article}

 
\begin{document}

\vbadness=10000 % Turns off the Under full \vbox warnings

\oddsidemargin=0.6in
\evensidemargin=0.7in
\setlength{\textwidth}{8in}
\setlength{\headsep}{0in}

 
\begin{titlepage}
  \vspace*{\stretch{55}}
  \begin{center}
   
   \large {\bf SCS-C: A C-language Implementation of a}\\
   \large {\bf Simple Classifier System}\\[10mm]

   \large {\bf K.P.Williams and S.A.Williams}\\[7mm]

   \large RUCS/98/TR999/X \\[40mm]
   \Large {\bf University of Reading}\\[3mm]
   \Large {\bf Department of Computer Science}\\[3mm]
   \Large {\bf Parallel, Emergent and Distributed}\\
   \Large {\bf Architectures Laboratory}\\
   \Large {\bf (PEDAL) Group}\\[5mm]
   \large \today\\
  \end{center}
 \vspace*{\stretch{2}}
 
\textheight=8in

\end{titlepage}

\oddsidemargin=0.7in
\textwidth=5.0in
\evensidemargin=0.4in
\marginparwidth=71pt


\title{\bf
   \large {\bf SCS-C: A C-language Implementation of a}\\
   \large {\bf Simple Classifier System}
       }
\author{
{\sc Kenneth P. Williams and Shirley A. Williams}   \\
Department of Computer Science, \\
University of Reading, \\
PO Box 225, Whiteknights, Reading, UK, RG6 6AY\\
}
 
\maketitle

\pagenumbering{arabic}

\setcounter{page}{1}

\textheight=8in

\section{Introduction}

  SCS-C\footnote{{\bf Disclaimer:} SCS-C is distributed
under the terms described in the file
\verb!NOWARRANTY!. These terms are taken from the GNU
Public License. This means that SCS-C has no warranty
implied or given, and that the authors assume no
liability for damage resulting from its use or misuse.}
is a pure ANSI C-language translation of the original Pascal 
SCS code presented by Goldberg\cite{1}. It is a direct
translation and contains no enhancements as such, it's
operation is essentially the same as that of the
original, Pascal version. The report is included as a
concise introduction to the SCS-C distribution. It is
presented with the assumptions that the reader has a
general understanding of Goldberg's original Pascal
code, and a good working knowledge of the C
programming language. The report begins with an outline
of the files included in the SCS-C distribution,
followed by a discussion of the points of SCS-C where
it differs from those of the Pascal version.  
 

\section{Files Distributed with SCS-C}

  The following is a list of files distributed with
SCS-C, the routines contained within those files and
the simple \verb!#include! structure of the SCS-C
distribution. The source files that make up SCS-C essentially
correspond one-to-one with those presented in the Pascal
version, with the addition of a single header file
\verb!scs.h! as described below.

\begin{description}

\item[\verb!scs.h!] This is the only header file used in SCS-C
and is \verb!#include!'ed into all other source files. It
contains several \verb!#define! definitions, notably:
\verb!LINELENGTH!, which determines the
column width of printed output and can be set to any desired 
positive value. It also contains all \verb!typedef!'s, all
\verb!struct! definitions as well as prototypes of all
functions used in the source files.
 

\item[\verb!advance.c!] contains code to advance all variables
by one time step.
  \begin{description}
    \item[\verb!advance()!] advance winner record. 
  \end{description}

\item[\verb!aoc.c!] contains routines for the apportionment of
credit.
  \begin{description}
    \item[\verb!initaoc()!] initialize clearing house record.
    \item[\verb!initrepaoc()!] initial report of clearinghouse
parameters.
    \item[\verb!auction()!] auction among currently matched
classifiers - returns the winner.
    \item[\verb!clearinghouse()!] distribute payment from
recent winner to oldwinner. 
    \item[\verb!taxcollector()!] collect existence and bidding
taxes from population members. 
    \item[\verb!reportaoc()!] report who pays to whom. 
    \item[\verb!aoc()!] apportionment of credit coordinator.
  \end{description}



\item[\verb!detector.c!] convert environmental states to
environmental message.
  \begin{description}
    \item[\verb!detectors()!] convert environmental state to
environmental message.
    \item[\verb!writemessage()!] write message in bit-reverse
order.
    \item[\verb!reportdetectors()!] write out environmental
message.
    \item[\verb!initdetectors()!] dummy detector
initialization routine.
    \item[\verb!initrepdetectors()!] dummy initial detectors
report.
  \end{description}


\item[\verb!effector.c!] contains the single effector routine.
  \begin{description}
    \item[\verb!effector()!] set action in classifier output
as dictated by auction winner.
  \end{description}


\item[\verb!environ.c!] controls multiplexor environment.
  \begin{description}
    \item[\verb!generatesignal()!] generate random signal.
    \item[\verb!decode()!] decode substring into a single
unsigned binary value.
    \item[\verb!multiplexoroutput()!] calculate correct
output of the 6-input/1-output multiplexor function (described
in Goldberg \cite{1}) being learned by SCS-C.
    \item[\verb!initenvironment()!] initialize the multiplexor
environment.
    \item[\verb!initrepenvironment()!] write initial
environmental report. 
    \item[\verb!writesignal()!] write signal in bit-reverse
order.
    \item[\verb!reportenvironment()!] write current
multiplexor info.
  \end{description}


\item[\verb!ga.c!] contains genetic algorithm routines for
SCS.
  \begin{description}
    \item[\verb!initga()!] initialize ga parameters.
    \item[\verb!initrepga()!] initial ga report.
    \item[\verb!select()!] select a single individual
according to strength.
    \item[\verb!mutation()!] mutate a single position with
specified probability.
    \item[\verb!bmutation()!] mutate a single bit with
specified probability.
    \item[\verb!crossover()!] cross a pair at a given site
with mutation on the ternary bit transfer. 
    \item[\verb!worstofn()!] select worst individual from
random subpopulation of size n. 
    \item[\verb!matchcount()!] count number of positions of
similarity.
    \item[\verb!crowding()!] replacement using modified De
Jong crowding. 
    \item[\verb!statistics()!] contains population statistics
routines - computes max, avg, min, sum of strength. 
    \item[\verb!ga()!] coordinate selection, mating,
crossover, mutation and replacement. 
    \item[\verb!reportga()!] report on mating, crossover and
replacement.
  \end{description}




\item[\verb!initial.c!] contains routines to coordinate
program initialization.
  \begin{description}
    \item[\verb!initrepheader()!] write a report header to a
specified output file/device. 
    \item[\verb!interactiveheader()!] clear screen and print
interactive header.
    \item[\verb!safe\_malloc()!] memory allocation routine
which terminates program (with an error message) on failure.
    \item[\verb!initialization()!] coordinate input and
initialization.
    \item[\verb!repchar()!] repeatedly write a character to
stdout.
    \item[\verb!skip()!] skip lines.
  \end{description}

 

\item[\verb!io.c!] contains file input/output routines.
  \begin{description}
    \item[\verb!page()!] start a new page.
    \item[\verb!open\_input()!] read filenames in interactive
mode.
    \item[\verb!open\_output()!] read filenames in interactive
mode. 
  \end{description}


\item[\verb!perform.c!] classifier matching routines.
  \begin{description}
    \item[\verb!randomchar()!] set position at random with
specified generality probability.
    \item[\verb!readcondition()!] read a single classifier
condition. 
    \item[\verb!readclassifier()!] read a single classifier.
    \item[\verb!countspecificity()!] count condition
specificity.
    \item[\verb!initclassifiers()!] init population of
classifier.
    \item[\verb!initrepclassifiers()!] initial report on
population parameters. 
    \item[\verb!writecondition()!] convert internal condition
format to external format and write to file/device. 
    \item[\verb!writeclassifier()!] write a single
classifier.
    \item[\verb!reportclassifiers()!] generate classifiers
report.
    \item[\verb!match()!] match a single condition to a single
message.
    \item[\verb!matchclassifiers()!] compare all classifiers
against current environmental message and create a list of
matches. 
  \end{description}



\item[\verb!reinfor.c!] functions for reinforcement and
criterion.
  \begin{description}
    \item[\verb!initreinforcement()!] initialize reinforcement
parameters.
    \item[\verb!initrepreinforcement()!] initial reinforcement
report. 
    \item[\verb!criterion()!] return true if criterion is
achieved.  
    \item[\verb!payreward()!] pay reward to appropriate
individual.  
    \item[\verb!reportreinforcement()!] report award.
    \item[\verb!reinforcement()!] make a payment if criterion
is satisfied.
  \end{description}


\item[\verb!report.c!] report coordination functions.
  \begin{description}
    \item[\verb!reportheader()!] send report header to
specified file/device.
    \item[\verb!report()!] report coordination routine. 
    \item[\verb!consolereport()!] write report to console.
    \item[\verb!plotreport()!] write plot figures to file.
  \end{description}


\item[\verb!scs.c!] This contains the {\tt main} SCS function. 
  \begin{description}
    \item[\verb!main()!] 
  \end{description}

\item[\verb!timekeep.c!] contains the timekeeping routines.
  \begin{description}
    \item[\verb!addtime()!] increment iterations counter and
set carry flag if necessary.
    \item[\verb!inittimekeeper()!] initialize timekeeper.
    \item[\verb!initreptimekeeper()!] initialize timekeeper
report.
    \item[\verb!timekeeper()!] keep time and set flags for
time-driven events. 
    \item[\verb!reporttime()!] print out block and iteration
number. 
  \end{description}

 

\item[\verb!utility.c!] this file contains various utility
routines including the SCS {\tt halt()} routine, and also the 
random number utility functions, which include; 



  \begin{description}
    \item[\verb!randomperc()!] returns a single uniformly
distributed, real, pseudo-random number between 0 and 1 using
the subtractive method, as described by Knuth in \cite{2}. 
    \item[\verb!rnd(low, high)!] returns a uniformly
distributed integer between {\tt low} and {\tt high}.  
    \item[\verb!rndreal(low, high)!] returns a uniformly
distributed real between {\tt low} and {\tt high}.  
    \item[\verb!warmup\_random()!] primes the random number
generator.
    \item[\verb!advance\_random()!]  generates batches of 55
random numbers.
    \item[\verb!flip(p)!] flips a biased coin, returning 1
with probability {\tt p}, and 0 with probability {\tt 1-p}. 
    \item[\verb!halt()!] Terminates SCS-C through prompting
for user input.
    \item[\verb!initrandomnormaldeviate()!] initialization
routine for function {\tt randomnormaldeviate}.
    \item[\verb!noise(mu, sigma)!] generates a normal random
variable with mean {\tt mu} and standard deviation {\tt
sigma}.  
    \item[\verb!randomize()!] asks user for random number
seed.
    \item[\verb!randomnormaldeviate()!] is a utility used by
{\tt noise} - it computes a standard normal random variable.
  \end{description}



The following are the test data files. All these are identical to
those presented in Goldberg \cite{1}.

\item[\verb!environ.dta!] This is a sample {\it efile}.

\item[\verb!ga.dta!] This is a sample {\it gafile}.

\item[\verb!perfect.dta!] This is a sample {\it cfile} for the
perfect rule set experiments of Chapter 6. 

\item[\verb!lessthan.dta!] This is a less than perfect 
{\it cfile} for default  hierarchy experiments of Chapter 6.

\item[\verb!reinf.dta!] This is a sample {\it rfile}.

\item[\verb!time.dta!] This is a sample {\it tfile}.

\item[\verb!class100.dta!] This is a sample {\it cfile} for
the clean-slate experiments of Chapter 6. Only 10 of each
type of rule are shown for brevity.


\end{description}



\verb!Makefile! is a simple UNIX makefile for SCS-C.

\section{Implementation Features of SCS-C}

\subsection{Translation Issues and Memory Utilization}

  Several differences between the Pascal and ANSI-C
programming languages forced a number of minor changes in the
way SCS-C operates in comparison to the original SCS program.
Firstly, in SCS-C all arrays are 1 element larger than their
SCS originals. This is because in C all arrays are indexed
from 0 rather than 1. Loop bounds were left unchanged to avoid
the risk of introducing errors into the C implementation (due
to time constraints).

  Secondly, parameter passing in Pascal is achieved by
reference rather than by value as in C. Consequently, it
seemed easiest to make the variables in SCS-C dynamic and to
pass parameters to/from functions so as to achieve
call-by-reference parameter passing. This introduction of
dynamic variables will allow easier extension of SCS-C in
future experimentation since variables, such as population
sizes, can be more easily varied. 

  The introduction of dynamic variables also has the effect of
moving responsibility for memory management onto the
programmer. This can be controlled via the use of the {\tt
safe\_malloc()}  function (defined in {\tt initial.c}) and 
the {\it free} function (used in {\tt scs.c}).


\subsection{Input / Output}

  Most of the input {\it readln()} statements in Pascal were
translated into equivalent {\it scanf()} calls in SCS-C.
Program termination in the original SCS version however is
achieved in the {\tt halt()} function (defined in {\tt
utility.c}) via the use of Pascal {\tt kbhit()} function.
Since no equivalent function exists in ANSI-C the solution
implemented in SCS-C is to initially prompt the user to enter
a number of generations the genetic algorithm is to run for
and then simply query the user whether they wish the run to
continue or terminate. This leads to the following input
screen (see Fig.\ref{input}).

\begin{figure}
\begin{center}
\begin{verbatim}
        prompt% ./scs
        How many generations ? : 2000
         Enter random number seed, 0.0 to 1.0 -> 0.3333
 
 
 
        |------------------------|
             Iteration = 50
             P correct = 0.960000
           P50 correct = 0.960000
        |------------------------|
 
 
 
        |------------------------|
             Iteration = 100
             P correct = 0.980000
           P50 correct = 1.000000
        |------------------------|


            . . . . . . . . . . 

 
        |------------------------|
             Iteration = 2000
             P correct = 0.999000
           P50 correct = 1.000000
        |------------------------|
        Halt (y/n) ? >> y



\end{verbatim}
\caption{Example SCS-C Input Screen}
\label{input}
\end{center}
\end{figure}


  Output from SCS-C is essentially identical to that of the
original SCS implementation. Two files ({\tt plot.out} and
{\tt rep.out}) are produced, sample outputs as in Figs.
(\ref{parameters1}, \ref{parameters2}, \ref{gens0}, \ref{gens2000}).


\begin{figure}
\begin{center}
\begin{verbatim}
   ----------------------------------------------------------- 
   |       SCS-C (v1.0) - A Simple Classifier System         | 
   |    (c) David E. Goldberg 1987, All Rights Reserved      | 
   |    C version by Kenneth P. Williams, U. of Reading      |
   -----------------------------------------------------------

 
   Population Parameters
   ---------------------
   Number of classifiers   = 10
   Number of positions   = 6
   Bid coefficient = 0.100000
   Bid spread = 0.075000
   Bidding tax = 0.010000
   Existence tax = 0.000000
   Generality probability = 0.500000
   Bid specificity base = 1.000000
   Bid specificity mult. = 0.000000
   Ebid specificity base = 1.000000
   Ebid specificity mult. = 0.000000
 

   Environmental Parameters (Multiplexor)
   --------------------------------------
   Number of Address Lines = 2
   Number of Data Lines = 4
   Total Number of Lines = 6



\end{verbatim}
\caption{SCS-C Parameters Report (cont...)}
\label{parameters1}
\end{center}
\end{figure}

 
\begin{figure}
\begin{center}
\begin{verbatim}
   Apportionment of Credit Parameters
   ----------------------------------
   Bucketbrigadeflag   =  False

 
   Reinforcement Parameters
   ------------------------
   Reinforcement reward  =  1.000000

 
   Timekeeper Parameters
   ---------------------
   Initial iteration         = 0
   Initial block  = 0
   Report period = 2000
   Console report period = 50
   Plot report period = 50
   GA period = -1

 
   Genetic Algorithm Parameters
   ----------------------------
   Proportion to select/gen  = 0.200000
   Number to select  = 1
   Mutation Probability = 0.020000
   Crossover Probability = 1.000000
   Crowding Factor = 3
   Crowding Subpopulation = 3


\end{verbatim}
\caption{SCS-C Parameters Report}
\label{parameters2}
\end{center}
\end{figure}



\begin{figure}
\begin{center}
\begin{verbatim}
   Snapshot report
   ---------------
 
   [ Block : Iteration] = [ 0 : 0 ]
 
 
   Current Multiplexor Status
   --------------------------
   Signal                =  0  0  0  0  0  0
   Decoded address       = 0
   Multiplexor output    = 0
   Classifier output     = 0
 
   Environmental message:   000000
 
     No.   Strength       bid   ebid    M   Classifier
   --------------------------------------------------
 
        1  10.00000  0.00000  0.00000    ###000: [ 0 ]
        2  10.00000  0.00000  0.00000    ###100: [ 1 ]
        3  10.00000  0.00000  0.00000    ##0#01: [ 0 ]
        4  10.00000  0.00000  0.00000    ##1#01: [ 1 ]
        5  10.00000  0.00000  0.00000    #0##10: [ 0 ]
        6  10.00000  0.00000  0.00000    #1##10: [ 1 ]
        7  10.00000  0.00000  0.00000    0###11: [ 0 ]
        8  10.00000  0.00000  0.00000    1###11: [ 1 ]
        9  10.00000  0.00000  0.00000    ######: [ 0 ]
       10  10.00000  0.00000  0.00000    ######: [ 1 ]
 
 
   New winner [1] : Old winner [1]
 
 
   Reinforcement Report
   --------------------
   Proportion correct (from start) = 0.000000
   Proportion correct (last 50) = 0.000000
   Last winning classifier number  = 0


\end{verbatim}
\caption{Report on SCS-C After 0 Generations}
\label{gens0}
\end{center}
\end{figure}
 


\begin{figure}
\begin{center}
\begin{verbatim}
   Snapshot report
   ---------------
 
   [ Block : Iteration] = [ 0 : 2000 ]
 
 
   Current Multiplexor Status
   --------------------------
   Signal                =  0  0  1  1  0  1
   Decoded address       = 1
   Multiplexor output    = 1
   Classifier output     = 1
 
   Environmental message:   001101
 
     No.   Strength       bid   ebid    M   Classifier
   --------------------------------------------------
 
        1  9.09091  0.90909  1.03956    ###000: [ 0 ]
        2  9.09091  0.90909  0.82836    ###100: [ 1 ]
        3  9.09091  0.90909  0.91054    ##0#01: [ 0 ]
        4  9.09091  0.90909  0.96905  X ##1#01: [ 1 ]
        5  9.09091  0.90909  0.84879    #0##10: [ 0 ]
        6  9.09091  0.90909  0.97096    #1##10: [ 1 ]
        7  9.09091  0.90909  0.95323    0###11: [ 0 ]
        8  9.09091  0.90909  0.95478    1###11: [ 1 ]
        9  0.00000  0.00000  0.06038  X ######: [ 0 ]
       10  0.00000  0.00000  0.02542  X ######: [ 1 ]
 
 
   New winner [4] : Old winner [5]
 
 
   Reinforcement Report
   --------------------
   Proportion correct (from start) = 0.999000
   Proportion correct (last 50) = 1.000000
   Last winning classifier number  = 4


\end{verbatim}
\caption{Report on SCS-C After 2000 Generations}
\label{gens2000}
\end{center}
\end{figure}
 
 The SCS code may be easily adapted to operate in interactive
mode (as indicated at points in {\tt initialize.c} and {\tt
scs.c}).

\section{Final Comments}

  SCS-C is intended to be a simple program for first
time classifier system experimentation. It is not
intended to be definitive in terms of its efficiency
or the grace of its implementation. As such it's release may
be seen as complimentary to the distribution of the C
implementation of the Simple Genetic Algorithm ({\tt SGA-C})
by Smith, Goldberg and Earickson in \cite{3}. 

  The authors are interested in the comments, criticisms, 
and bug reports from SCS-C users, so that the code can be 
refined for easier use in subsequent versions. Please email 
your comments to {\bf K.P.Williams@reading.ac.uk}, or write to:

\begin{center} 
Ken Williams,   \\
University of Reading,             \\
Department of Computer Science,    \\
Parallel, Emergent and Distributed\\
Architectures Laboratory (PEDAL),\\
PO Box 225, Whiteknights, Reading,\\
Berkshire, ENGLAND RG6 6AY\\
\end{center} 
 

\subsection*{Acknowledgments}

  Ken Williams gratefully acknowledges the support of the
Engineering and Physical Sciences Research Council (EPSRC)
of Great Britain under grant number 94701308, and also of support
from the Department of Computer Science, University of Reading.  

\newpage

\begin{thebibliography}{99}

\bibitem[1]{1} Goldberg, D.E., {\it Genetic Algorithms
in Search Optimization and Machine Learning},
Addison-Wesley, Reading: MA, (1989). 

\bibitem[2]{2} Knuth, D.E., {\it The Art of Computer
Programming}, Vol. 2, (2nd Edition), Addison-Wesley, Reading:
MA, (1981).

\bibitem[3]{3} Smith R.E, Goldberg D.E., and
Earickson, J.A, {\it  SGA-C: A C-language
Implementation of a Simple Genetic Algorithm},
TCGA Report No. 91002, The Clearing House for Genetic
Algorithms, Dept. of Engineering Mechanics, U. of Alabama, 
Tuscaloosa, AL 35487, (1994).

\end{thebibliography}

\end{document}


